% Compiled with LuaLaTeX
% TeX-command-extra-options: "-shell-escape"
\documentclass[convert={outext=.png},border=10pt]{standalone}
\usepackage{fontspec}
\setmainfont{Roboto}
\usepackage{pgfplots}
\pgfplotsset{compat=newest}
\usepackage{luacode}

\definecolor{fg}{RGB}{150,150,150} % Foreground color

\begin{document}
%
\pgfplotsset{title style={xshift=1cm,yshift=-0.2cm}}%
%
\pgfmathdeclarefunction{lg10}{1}{%
    \pgfmathparse{ln(#1)/ln(10)}%
}%
%
% Load utils.lua and config.lua.
\begin{luacode*}
  -- utils.lua contains lua functions to be called to be called in this file.
  myutils = require("utils")
  -- config.lua contains the user input parameters.
  myconfig = require("config")
\end{luacode*}
%
% Define a LaTeX command that generates one heatmap. Inside the command, there
% are various macros that define style options and data paths, such as,
% "\varxtick", "\varsize" or "\vardatait". These macros should be initialized 
% before calling the \heatmaps command.
\newcommand{\heatmap}
{%
  \begin{axis}[
    enlarge x limits={abs=0.5},
    enlarge y limits={abs=0.},
    grid=minor,
    tick style={draw=none}, % Hide the ticks
    minor grid style={fg,very thin},
    axis on top,
    % x-axis style
    minor xtick={1.5,2.5,...,16.5},
    x label style={font=\normalsize},
    xtick=\varxtick, % Recover x-ticks positions from macro
    xticklabel={
     \if\varxticklabel 1
       \pgfmathparse{\tick-1}
       $10^{\pgfmathprintnumber{\pgfmathresult}}$
     \fi
    },
    x tick label style={font=\small},
    xlabel=\varxlabel, % Recover the x-axis label from macro
    % y-axis style
    minor ytick={1.5,2.5,...,16.5},
    y label style={at={(axis description cs:-0.15,0.5)},rotate=0,font=\normalsize},
    ytick={\varytick}, % Recover y-ticks positions from macro
    yticklabel={
     \if\varyticklabel 1
       \pgfmathparse{\tick-1}
       $10^{\pgfmathprintnumber{\pgfmathresult}}$
     \fi
    },
    y tick label style={font=\small,rotate=90},
    ylabel=\varylabel, % Recover the y-axis label from macro
    mesh/ordering=y varies, 
    unbounded coords=jump,
    height=\varsize\linewidth, % Set the size (height and width) from macro
    width=\varsize\linewidth,
    colormap={whiteblue}{color={black} color=(white)},
    point meta min=0,
    point meta max=4,
    view={0}{90},
    scale only axis=true, % Enforce the paramaters height and width to describe
                          % the size of the axis without including the 
                          % decorations
    title=\textsc{\vartitle}, % Recover the name of the heatmap from macro
    at={(\varcoordx,\varcoordy)}, % Recover the position of the heatmap from
                                  % macro
  ]%
  \addplot[matrix plot*,mesh/cols=17,point meta=explicit]
      table[meta expr=lg10(\thisrow{it}),col sep=comma]
      {\vardatait}; % Obtain the data path from macro
  \addplot[white,only marks,mark=*,mark size=0.75pt] 
      table[col sep=comma] {\vardatabounds};
  \end{axis}
}%
%
% Define a LaTeX command that generates the colorbar. Because all the heatmaps
% of the grid share the same colorbar, we need to define the colorbar as a
% standalone, outside of the heatmap plots. To do so we use the 
% `\pgfplotscolorbardrawstandalone[...].`
\newcommand{\heatcolorbar}
{%
  \pgfplotscolorbardrawstandalone[
    colormap={whiteblue}{color={black} color=(white)},
    point meta min=0,
    point meta max=3.68,
    colorbar horizontal,
    colorbar style={
       ylabel={\#it},
       yticklabel pos=upper,
       ylabel style={rotate=-90, xshift=.22cm},
       xtick={0,1,1.7,2,2.70,3,3.68},
       xticklabels={$1e0$, $1e1$, $5e1$, $1e2$, $5e2$, $1e3$, $5e3$},
       x tick label style={font=\footnotesize},
       xticklabel pos=upper,
       at={(\varcbarx\linewidth,\varcbary\linewidth)}, 
       width=\varcbarsizew\linewidth,
       anchor=center, % The position of the colorbar defined with `at` is now
       % based on the center of the figure rather than the left bottom corner
    },
    colorbar/width=\varcbarsizeh\linewidth,
    colormap access=map,
  ]%
}%
%
\begin{tikzpicture}[fg]
  \begin{luacode*}

    -- Get the grid information based on the user input parameters in
    -- config.lua and computed from the function `get_grid_heatmaps(...)`
    -- loaded from utils.lua.
    grid = myutils.get_grid_heatmaps(#myconfig.plots, 
                                     myconfig.gridcols,
                                     myconfig.relativesize)

    -- Loop over the heatmaps
    for i = 1, #myconfig.plots
    do
      
      -- Extract name of the heatmap and the source of the data from config.lua
      title   = myconfig.plots[i][2]
      pathcsv_it = myconfig.dircsv .. "/iterations/" ..  myconfig.plots[i][1] 
        .. ".csv"
      pathcsv_bounds = myconfig.dircsv .. "/bounds/" .. myconfig.plots[i][1] 
        .. ".csv"
      
      -- Set the LaTeX macros that will define the style of the current heatmap
      tex.print("\\def\\vardatait{"       .. pathcsv_it             .. "}")
      tex.print("\\def\\vardatabounds{"   .. pathcsv_bounds         .. "}")
      tex.print("\\def\\vartitle{"        .. title                  .. "}")
      tex.print("\\def\\varcoordx{"       .. grid.coordx[i]         .. "}")
      tex.print("\\def\\varcoordy{"       .. grid.coordy[i]         .. "}")
      tex.print("\\def\\varsize{"         .. myconfig.relativesize  .. "}")
      tex.print("\\def\\varxtick{"        .. grid.xtick[i]          .. "}")
      tex.print("\\def\\varxticklabel{"   .. grid.xticklabel[i]     .. "}")
      tex.print("\\def\\varxlabel{"       .. grid.xlabel[i]         .. "}")
      tex.print("\\def\\varytick{"        .. grid.ytick[i]          .. "}")
      tex.print("\\def\\varyticklabel{"   .. grid.yticklabel[i]     .. "}")
      tex.print("\\def\\varylabel{"       .. grid.ylabel[i]         .. "}")

      -- Generate the current heatmap
      tex.print("\\heatmap")
    end

    -- Set the LaTeX macros that will define the style of the colorbar
    tex.print("\\def\\varcbarx{"      .. grid.cbarx     .. "}")
    tex.print("\\def\\varcbary{"      .. grid.cbary     .. "}")
    tex.print("\\def\\varcbarsizew{"  .. grid.cbarsizew .. "}")
    tex.print("\\def\\varcbarsizeh{"  .. grid.cbarsizeh .. "}")

    -- Generate the colorbar
    tex.print("\\heatcolorbar")
  \end{luacode*}
\end{tikzpicture}%
\end{document}

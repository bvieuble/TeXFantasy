%%% Compiled with XeLaTeX
%%% TeX-command-extra-options: "-shell-escape"
\documentclass[convert={outext=.png},border=10pt]{standalone}
\usepackage{fontspec}
\setmainfont{Roboto}
\usepackage{tikz}

\usetikzlibrary{matrix, positioning,fit,backgrounds,arrows.meta}

\definecolor{myblue}{RGB}{25,130,196}
\definecolor{myblue2}{RGB}{25,50,220}
\definecolor{mygreen}{RGB}{100,180,28}
\definecolor{myyellow}{RGB}{255,202,58}
\definecolor{myorange}{RGB}{255,154,92}
\definecolor{myred}{RGB}{255,92,97}
\definecolor{mypurple}{RGB}{116,83,162}
\definecolor{fg}{RGB}{150,150,150}
\definecolor{mygray}{RGB}{190,190,190}

\colorlet{myblue-mild}{myblue!60}
\colorlet{myblue2-mild}{myblue2!60}
\colorlet{mygreen-mild}{mygreen!60}
\colorlet{myyellow-mild}{myyellow!60}
\colorlet{myorange-mild}{myorange!60}
\colorlet{myred-mild}{myred!60}
\colorlet{mypurple-mild}{mypurple!60}
\colorlet{mygray-mild}{mygray!60}


\begin{document}

\begin{tikzpicture}[>=stealth,thick,fg,baseline]
    \matrix [matrix of math nodes,left delimiter={[},right delimiter={]}, 
    column sep=1mm, row sep=1mm,nodes={minimum width=.7cm,text depth=.14cm,
    text height=.3cm}](A){ 
    \phantom{a_{1,1}} & \phantom{a_{1,2}} & \phantom{a_{1,k}} & & & 
    \phantom{a_{1,6}} \\
    \phantom{a_{2,1}} & \phantom{a_{2,2}} & \phantom{a_{2,k}} & & & 
    \phantom{a_{2,n}} \\  
    \phantom{a_{k,1}} & \phantom{a_{k,2}} & \color{black} a_{k,k} & & & \phantom{a_{k,n}} \\
     & & & & & \\
    \phantom{a_{j,1}} & & \color{black} a_{j,k} & & & \phantom{a_{j,n}} \\
    \phantom{a_{n,1}} & \phantom{a_{n,2}} & \phantom{a_{n,3}} & 
    \phantom{a_{n,j}} & & \\
    };

    \node[right=15pt of A-3-6] (l1) {};
    \node[right=15pt of A-5-6] (l2) {};

    \node (L) at (A-2-1.east) {\Large \color{black} L};
    \node (U) at (A-1-2.south) {\Large \color{black} U};

    \node[above=5pt of A-1-3] (step) {\footnotesize Step $k$};

    \node[fit=(A-3-1)(A-3-6),dashed,inner sep=-3pt,rounded corners=5pt,draw] 
        {};
    \node[fit=(A-5-1)(A-5-6),dashed,inner sep=-3pt,rounded corners=5pt,draw] 
        {};

    \path[<->] (l1.west) edge[bend left=55] node[name=max,right,near end] 
        {\footnotesize$|a_{j_k}|=\max\limits_{l\in\{k,n\}}|a_{l,k}|$} 
        (l2.west);
    \path[->] (step) edge (A-3-3);

    \begin{scope}[on background layer]
        \draw[fill=myblue-light,draw=fg] (A-1-1.north west) -- (A-6-1.south west) -- 
            (A-6-2.south east) -- (A-2-2.south east) -- cycle;
        \draw[fill=myyellow-light,draw=fg] (A-1-1.north west) -- (A-1-6.north east) --
            (A-2-6.south east) -- (A-2-2.south east) -- cycle;
        \node[fit=(A-3-3)(A-6-3),inner sep=0pt,fill=myred-light,
              rounded corners=5pt] {};
    \end{scope}

    \node[below=10pt of A-6-3] (title) {\large Partial pivoting};
\end{tikzpicture}

\begin{tikzpicture}[>=stealth,thick,fg,baseline]
    \matrix [matrix of math nodes,left delimiter={[},right delimiter={]}, 
    column sep=1mm, row sep=1mm,nodes={minimum width=.7cm,text depth=.14cm,
    text height=.3cm}](A){ 
    \phantom{a_{1,1}} & \phantom{a_{1,2}} & & & & \phantom{a_{1,6}} \\
    \phantom{a_{2,1}} & \phantom{a_{2,2}} & \phantom{a_{2,k}} & & & 
    \phantom{a_{2,n}} \\  
    \phantom{a_{k,1}} & \phantom{a_{k,2}} & \color{black} \epsilon_{\mbox{\tiny 
    \textsc{stc}}} & & & \phantom{a_{k,n}} \\
     & & & & & \\
    \phantom{a_{j,1}} & & \vdots & & & \phantom{a_{j,n}} \\
    \phantom{a_{n,1}} & \phantom{a_{n,2}} & \phantom{a_{n,3}} &
    \phantom{a_{n,j}} & & \\
    };

    \node[right=15pt of A-3-6] (l1) {};
    \node[right=15pt of A-5-6] (l2) {};

    \node[above=5pt of A-1-3] (step) {\footnotesize Step $k$};
    \node[right=7pt of A-3-3] (tau) {\footnotesize if $|a_{k,k}|<
    \epsilon_{\mbox{\tiny \textsc{stc}}}$};

    \node (L) at (A-2-1.east) {\Large \color{black} L};
    \node (U) at (A-1-2.south) {\Large \color{black} U};

    \path[->] (tau) edge (A-3-3);
    \path[->] (step) edge (A-3-3);

    \begin{scope}[on background layer]
        \draw[fill=myblue-light,draw=fg] (A-1-1.north west) -- (A-6-1.south west) -- 
            (A-6-2.south east) -- (A-2-2.south east) -- cycle;
        \draw[fill=myyellow-light,draw=fg] (A-1-1.north west) -- (A-1-6.north east) -- 
            (A-2-6.south east) -- (A-2-2.south east) -- cycle;
        \node[fit=(A-3-3)(A-6-3),inner sep=0pt,fill=myred-light,
              rounded corners=5pt] {};
    \end{scope}

    \node[below=10pt of A-6-3] (title) {\large Static pivoting};
\end{tikzpicture}

\end{document}
